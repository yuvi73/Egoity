\documentclass[conference]{IEEEtran}
\usepackage[utf8]{inputenc}
\usepackage{graphicx}
\usepackage{amsmath}
\usepackage{booktabs}
\usepackage{multirow}
\usepackage{url}
\usepackage{hyperref}
\usepackage{cite}

\title{Development of Self-Recognition in Artificial Agents: A Computational Model of Emergent Self-Awareness}

\author{
    \IEEEauthorblockN{Your Name}
    \IEEEauthorblockA{
        Department of Computer Science\\
        Your University\\
        City, Country\\
        email@example.com
    }
}

\begin{document}

\maketitle

\begin{abstract}
This paper presents a computational model for studying the emergence of self-recognition in artificial agents. We implement a multi-agent simulation environment where agents develop self-awareness through interaction with their environment and other agents. The model incorporates key components of self-awareness including self-consistency, agency, and emotional states. Our results demonstrate that self-recognition can emerge as a result of the agents' experiences and interactions, with measurable metrics showing the development of self-awareness over time. The findings provide insights into the computational requirements for artificial self-awareness and have implications for the development of more sophisticated AI systems.
\end{abstract}

\begin{IEEEkeywords}
Self-awareness, artificial intelligence, multi-agent systems, computational modeling, self-recognition
\end{IEEEkeywords}

\section{Introduction}
The development of self-awareness in artificial systems remains a significant challenge in AI research. While current AI systems excel at specific tasks, they lack the fundamental self-awareness that characterizes human cognition. This paper presents a computational model that explores how self-recognition might emerge in artificial agents through interaction and experience.

Our model builds upon previous work in artificial consciousness \cite{cite1} and self-aware computing systems \cite{cite2}. We extend these approaches by incorporating emotional states and social interactions as key components in the development of self-awareness.

\section{Related Work}
Previous research in artificial self-awareness has focused on various aspects of the problem. Some approaches have emphasized the role of self-models \cite{cite3}, while others have explored the importance of social interaction \cite{cite4}. Our work integrates these perspectives into a unified computational model.

\section{Model Architecture}
\subsection{Agent Design}
Each agent in our simulation consists of several key components:

\begin{itemize}
    \item Self-Model: Maintains beliefs about the agent's own state and capabilities
    \item Memory System: Stores and processes experiences
    \item Emotional State: Tracks fear, joy, pride, and stress levels
    \item Action Selection: Decides actions based on current goals and emotional state
\end{itemize}

\subsection{Environment}
The simulation environment is a 2D grid where agents can:
\begin{itemize}
    \item Move in four directions
    \item Interact with other agents
    \item Find and consume resources
    \item Avoid dangers
\end{itemize}

\subsection{Self-Recognition Metrics}
We measure self-recognition through several metrics:
\begin{equation}
    \text{Self-Recognition} = \frac{\text{Memory Factor} + \text{Agency} + \text{Self-Consistency}}{3}
\end{equation}

where:
\begin{itemize}
    \item Memory Factor: Ratio of self-generated memories to total memories
    \item Agency: Measure of the agent's control over its actions
    \item Self-Consistency: Coherence of the agent's beliefs and actions
\end{itemize}

\section{Experimental Setup}
We conducted experiments with the following parameters:
\begin{itemize}
    \item Grid Size: 20x20
    \item Number of Agents: 5
    \item Simulation Steps: 1000
    \item Number of Simulations: 5
\end{itemize}

\section{Results}
Our experiments demonstrate the emergence of self-recognition in artificial agents. Figure \ref{fig:metrics} shows the development of key metrics over time.

\begin{figure}[!t]
    \centering
    \includegraphics[width=\columnwidth]{figures/metrics_development}
    \caption{Development of self-awareness metrics over time}
    \label{fig:metrics}
\end{figure}

Table \ref{tab:results} summarizes the average values of key metrics across all simulations.

\begin{table}[!t]
    \caption{Average Metric Values Across Simulations}
    \label{tab:results}
    \centering
    \begin{tabular}{lcc}
        \toprule
        Metric & Initial Value & Final Value \\
        \midrule
        Self-Consistency & 1.0000 & 1.0000 \\
        Agency & 0.5000 & 0.7500 \\
        Self-Recognition & 0.5333 & 0.8333 \\
        Memory Count & 2.2000 & 9.0000 \\
        \bottomrule
    \end{tabular}
\end{table}

\section{Discussion}
The results show that self-recognition can emerge in artificial agents through interaction and experience. Key findings include:

\begin{itemize}
    \item Self-consistency remains high throughout the simulation
    \item Agency increases as agents learn from their actions
    \item Self-recognition develops gradually over time
    \item Memory accumulation correlates with self-recognition development
\end{itemize}

\section{Conclusion}
Our computational model demonstrates that self-recognition can emerge in artificial agents through interaction and experience. The results suggest that self-awareness in artificial systems may develop through similar mechanisms as in biological systems, though further research is needed to explore this connection.

\section*{Acknowledgment}
The authors would like to thank the anonymous reviewers for their valuable feedback.

\begin{thebibliography}{1}
\bibitem{cite1}
A. Turing, "Computing Machinery and Intelligence," Mind, vol. 59, no. 236, pp. 433-460, 1950.

\bibitem{cite2}
J. McCarthy, "Programs with Common Sense," in Proceedings of the Teddington Conference on the Mechanization of Thought Processes, 1959.

\bibitem{cite3}
R. Brooks, "Intelligence Without Representation," Artificial Intelligence, vol. 47, pp. 139-159, 1991.

\bibitem{cite4}
M. Minsky, "The Society of Mind," Simon and Schuster, 1986.
\end{thebibliography}

\end{document} 